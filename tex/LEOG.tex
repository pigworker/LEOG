\documentclass{article}
\usepackage{a4}
\usepackage{pig}
\usepackage{latexsym}

\newcommand{\lsq}{\texttt{\symbol{91}}}
\newcommand{\bsl}{\texttt{\symbol{92}}}
\newcommand{\rsq}{\texttt{\symbol{93}}}
\newcommand{\lcu}{\texttt{\symbol{123}}}
\newcommand{\lba}{\texttt{\symbol{124}}}
\newcommand{\rcu}{\texttt{\symbol{125}}}

\begin{document}
\title{LEOG}
\author{Conor Mc Bride}
\maketitle

\section{introduction}

I'm trying to cook up a variant of the Calculus of (Inductive, etc) Constructions which is \emph{bidirectional} in Curry style, and \emph{observational} in that equality reflects what you can do with things, rather than how they are constructed (which makes a difference when the elimination form does not permit observation of construction).

The purpose of this note is simply to record what I think the rules are.


\section{syntax}

\newcommand{\Prop}{\textbf{Prop}}
\newcommand{\Type}[1]{\textbf{Type}_{#1}}
\newcommand{\TYPE}{\star}

\subsection{sorts}

We have sorts $\Prop$ and $\Type{n}$ for natural $n$. Administratively, we have a `topsort' $\TYPE$, which should appear only as the type in a checking judgement (i.e., morally, there is a separate type formation judgement) and as the upper bound in a cumulativity judgement (because a type of any sort is a type).

As in CIC, the sort $\Prop$ admits impredicative universal quantification.


\subsection{canonical things}

\newcommand{\A}[1]{\texttt{'#1}}
\newcommand{\nil}{\lsq\rsq}
\newcommand{\X}[2]{\lsq#1\lba#2\rsq}
\newcommand{\B}[2]{\bsl#1\texttt{.}\,#2}
\newcommand{\E}[2]{#1\lcu#2\rcu}
\newcommand{\PI}[3]{\lsq\A{Pi}\;#1\;\B{#2}{#3}\rsq}

We have \emph{atoms} $\A{\textit{atom}}$. We have pairs $\X st$, with the usual LISP right-bias convention that $\nil$ the syntacitcally valid name for the atom $\A{}$, and $\lba\lsq$ may be omitted along with the matching $\rsq$.

Invariance up to alpha-equivalence is quite helpful, so let us not use magical atoms for abstraction. Instead, let us have $\B xt$ abstracting the \emph{variable} $x$ from $t$.

Given these ingredients, we can construct canonical forms such as $\PI SxT$. Note that canonical forms are not expected to make sense in and of themselves: the best we can hope is that sense is made of them. They will always be type\emph{checked}.

Of course, once we have free variables, we have \emph{noncanonical} things. They embed in the canonical things as $\E eq$, where $e$ is noncanonical and $q$ is a canonical proof that $e$ fits where it's put. We may shorten $\E e\nil$ to $e$, as $\nil$ will always serve as the proof term when judgemental cumulativity/equality is enough.


\subsection{noncanonical things}

\newcommand{\e}[1]{\texttt{-}#1}
\newcommand{\R}[2]{#1\texttt{:}#2}

We have variables $x$ within $t$ for some $\B xt$. We also have $e\e s$ for noncanonical $s$, an eliminator appropriate to the type of $e$ (e.g., if $e$ is a function, $s$ is its argument).

Finally, we have $\R tT$, a \emph{radical}, allowing us to annotate a canonical term with a canonical type, and thus form redexes.


\section{typing awaiting computation}

\newcommand{\ch}[2]{#1 \:\ni\: #2}
\newcommand{\sy}[2]{#1 \:\in\: #2}
\newcommand{\hy}{\;\vdash\;}
\newcommand{\CU}[2]{\lsq\A{Le}\:#1\:#2\rsq}

We shall have judgement forms for checking $\ch Tt$ (\emph{relying} on $\ch\TYPE T$) and synthesis $\sy eS$ (guaranteeing that $\ch\TYPE S$). At some point, these will be closed appropriately under computation, but let's leave that to the next section.


\subsection{type checking}

For sorts $w$, $u$, we have

\[
\Rule{w > u} {\ch wu}\qquad
\Axiom{\TYPE > \Type n}\quad
\Axiom{\TYPE > \Prop}\quad
\Rule{m > n}{\Type m > \Type n}\quad
\Axiom{\Type n > \Prop}
\]
I'm aware that the Coq tradition (from the days of impredicative \textbf{Set}) is to place $\Prop$ on a par with $\Type 0$ and below $\Type 1$, but the above formulation says yes to all those things and more. Besides, this system isn't called LEOG for nothing.

Meanwhile, for function types,
\[
\Rule{\ch wS \hg \sy xS \hy \ch uT}
     {\ch u \PI SxT}
     \;\mbox{where}\;
     w = \left\{\begin{array}{ll}\TYPE & \mbox{if}\;u=\Prop\\
                                     u & \mbox{otherwise}\end{array}\right.
\]
making $\Prop$ impredicative.

For functions, we have
\[\Rule{\sy xS \hy \ch Tt}{\ch{\PI SxT}{\B xt}}
\]

On the checking side, that leaves us with only the shipment of synthesizable terms.
\[
\Rule{e\in S \hg \CU ST \ni q}
     {T\ni \E eq }
\]
where this $\A{Le}$ thing is \emph{propositional cumulativity}. For all sorts $u$,
because for $u=\Prop$
\[
\Rule{\TYPE\ni S \hg \TYPE\ni T}
     {u\ni\CU ST}
\]
Let us have
\[
\Axiom{\CU\Prop{\Type n} \ni \nil}\qquad
\Rule{i < j}
     {\CU{\Type i}{\Type j}\ni \nil}
\]
and the (domain contravariant!)
\[
\Rule{\CU{S'}S\ni \nil \hg \sy x{S'}\hy \CU T{T'}\ni \nil}
{\CU{\PI SxT}{\PI{S'}x{T'}}\ni\nil}
\]
along with the inclusion of judgemental equality
\[
\Rule{\TYPE\ni S\equiv T}
{\CU ST \ni \nil}
\]
Nontrivial proofs of cumulativity will arrive in due course.


\subsection{type synthesis}

I write a reverse turnstile to indicate an appeal to a local hypothesis,
which is as close as I get to mentioning the context. Radicals are
straightforward, thanks to the `topsort'.
\[
\Rule{\dashv\; \sy xS}
{\sy x S}
\qquad
\Rule{\ch\TYPE T\hg \ch Tt}
     {\sy{\R tT}T}
\qquad
\Rule{\sy e{\PI SxT}\hg \ch Ss}
{\sy{e\e s}{T[\R sS]}}
\]
Application is unremarkable,
except to note that (i) there is no need to mention $x$ by name when
substituting it, because it is clear that it is $x$ which has moved out of
scope, and (ii) that $s$ has to be radicalised before it can be substituted
for $x$, from sheer syntactic compatibility and to create new redexes.


\section{computation, na\"\i vely}

\newcommand{\st}{\leadsto}

The basic plan is to extend checking and synthesis with small step reduction, so
\[
\ch T{t\st v} \qquad \sy{e\st u}S
\]
with single substitution doing the work. In practice, I use a big step environment
machine, but let us at least have an executable specification. We should first of all plug
computation into typing.
\[
\Rule{\ch\TYPE{T\st V}\hg V\ni t}
     {\ch Tt}
\qquad
\Rule{\sy eS \hg \ch\TYPE{S\st V}}
     {\sy eV}
\]
Doing so establishes the obligation that $\st^\ast$ does enough computation to expose
a head canonical form, but need not go all the way to the canonical representative
of the judgemental equivalence class. The latter is certainly an option, however.

Note that these computation judgements have no \emph{subjects}. They may presume
that the thing to be reduced has been checked already. Observe that, above left,
the rule's client is responsible for $T$ being a type, whilst above right, $S$
the rule's first premise is responsible for $S$ being a type. Of course, the
computation rules are responsible for the reduced terms they output, which must
serve in place of the input. For checking, that means the type which checked the
input checks the output; for synthesis, that means the type synthesized for the
output must be \emph{possible} for the input --- type synthesis now admits
post-computation and is thus ambiguous.

So far, we have two rules which matter
\[
\Axiom{\sy {(\R{\B xt}{\PI SxT})\e s \st (\R{t}{T})[\R sS]}{T[\R sS]}}
\qquad
\Axiom{\ch T{\E{\R sS}\nil \st s}}
\]
together with a bunch of structural closure rules, e.g.
\[
\Rule{\sy xS \hy \ch\TYPE{T \st T'}}
     {\ch\TYPE{\PI SxT \st \PI Sx{T'}}}
\]
and a problem, if we want $\CU ST$ to be judgementally proof-irrelevant. We
can't very well claim that the proof doesn't matter if we keep checking to see
if it's $\nil$. We should, instead, check if it \emph{could} be $\nil$, and so
we shall, in due course.


\section{judgemental equality}

We shall have judgements for type-directed equality of chceckables and type
reconstructing equality of synthesizables.
\[
\ch T{t_0\equiv t_1} \qquad \sy{e_0\equiv e_1}{S}
\]
where the former presumes $\ch\TYPE T$, $\ch T{t_0}$ and $\ch T{t_1}$, while
the latter presumes $\sy{e_0}{S_0}$, $\sy{e_1}{S_1}$ and guarantees that
$\ch\TYPE S$, with $\ch\TYPE S{S_0}$ and $\ch\TYPE S{S_1}$.

Of course, these judgements are closed under computation in the appropriate way:
\[
\Rule{\ch\TYPE{T\st V}\hg \ch V{t_0\equiv t_1}}
     {\ch T{t_0\equiv t_1}}
\]
\[
\Rule{\ch T{t_0\st v_0}\hg \ch T{v_0\equiv t_1}}
     {\ch T{t_0\equiv t_1}}
     \qquad
\Rule{\ch T{t_1\st v_1}\hg \ch T{t_0\equiv v_1}}
     {\ch T{t_0\equiv t_1}}
\]
\[
\Rule{\sy{e_0\st w_0}{S_0}\hg \sy{w_0\equiv e_1}S}
     {\sy{e_0\equiv e_1}S}
     \qquad
\Rule{\sy{e_1\st w_1}{S_1}\hg \sy{e_0\equiv w_1}S}
     {\sy{e_0\equiv e_1}S}
\]
\[
\Rule{\sy{e_0\equiv e_1}S\hg \ch\TYPE{S\st V}}
     {\sy{e_0\equiv e_1}V}
\]

Now, for canonical things, we have only reflexivity.
\[
\Axiom{\ch w{\Prop\equiv\Prop}}\qquad
\Axiom{\ch w{\Type i\equiv\Type i}}
\]
For functions, we $\eta$-expand:
\[
\Rule{\sy xS \hy \ch T (\R{v_0}{\PI SxT})\e x \equiv (\R{v_1}{\PI SxT})\e x}
     {\ch{\PI SxT}{v_0\equiv v_1}}
\]
For embedded neutrals, we have
\[
\Rule{\sy{e_0\equiv e_1}S}
     {\ch T{\E{e_0}{q_0} \equiv \E{e_1}{q_1}}}
\]
studiously ignoring the proofs $q_0$ and $q_1$.

In the other direction, we have
\[
\Rule{\dashv\; \sy xS}
     {\sy{x\equiv x}S}
     \qquad
\Rule{\sy{e_0\equiv e_1}{\PI SxT}\hg \ch S{s_0\equiv s_1}}
     {\sy{e_0\e{s_0}\equiv e_1\e{s_1}}{T[\R{s_0}S]}}
\]

What to do about radicals that's better than
\[
\Rule{\ch\TYPE{T_0\equiv T_1}\hg \ch{T_0}{t_0\equiv t_1}}
{\sy{\R{t_0}{T_0}\equiv\R{t_1}{T_1}}{T_0}}
\]
is something I'll sleep on.

\end{document}
