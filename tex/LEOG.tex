\documentclass{article}
\usepackage{a4}
\usepackage{pig}

\newcommand{\lsq}{\texttt{\symbol{91}}}
\newcommand{\bsl}{\texttt{\symbol{92}}}
\newcommand{\rsq}{\texttt{\symbol{93}}}
\newcommand{\lcu}{\texttt{\symbol{123}}}
\newcommand{\lba}{\texttt{\symbol{124}}}
\newcommand{\rcu}{\texttt{\symbol{125}}}

\begin{document}
\title{LEOG}
\author{Conor Mc Bride}
\maketitle

\section{introduction}

I'm trying to cook up a variant of the Calculus of (Inductive, etc) Constructions which is \emph{bidirectional} in Curry style, and \emph{observational} in that equality reflects what you can do with things, rather than how they are constructed (which makes a difference when the elimination form does not permit observation of construction).

The purpose of this note is simply to record what I think the rules are.


\section{syntax}

\newcommand{\Prop}{\textbf{Prop}}
\newcommand{\Type}[1]{\textbf{Type}_{#1}}
\newcommand{\TYPE}{\star}

\subsection{sorts}

We have sorts $\Prop$ and $\Type{n}$ for natural $n$. Administratively, we have a `topsort' $\TYPE$, which should appear only as the type in a checking judgement (i.e., morally, there is a separate type formation judgement) and as the upper bound in a cumulativity judgement (because a type of any sort is a type).

As in CIC, the sort $\Prop$ admits impredicative universal quantification.


\subsection{canonical things}

\newcommand{\A}[1]{\texttt{'#1}}
\newcommand{\nil}{\lsq\rsq}
\newcommand{\X}[2]{\lsq#1\lba#2\rsq}
\newcommand{\B}[2]{\bsl#1\texttt{.}\,#2}
\newcommand{\E}[2]{#1\lcu#2\rcu}
\newcommand{\PI}[3]{\lsq\A{Pi}\;#1\;\B{#2}{#3}\rsq}

We have \emph{atoms} $\A{\textit{atom}}$. We have pairs $\X st$, with the usual LISP right-bias convention that $\nil$ the syntacitcally valid name for the atom $\A{}$, and $\lba\lsq$ may be omitted along with the matching $\rsq$.

Invariance up to alpha-equivalence is quite helpful, so let us not use magical atoms for abstraction. Instead, let us have $\B xt$ abstracting the \emph{variable} $x$ from $t$.

Given these ingredients, we can construct canonical forms such as $\PI SxT$. Note that canonical forms are not expected to make sense in and of themselves: the best we can hope is that sense is made of them. They will always be type\emph{checked}.

Of course, once we have free variables, we have \emph{noncanonical} things. They embed in the canonical things as $\E eq$, where $e$ is noncanonical and $q$ is a canonical proof that $e$ fits where it's put. We may shorten $\E e\nil$ to $e$, as $\nil$ will always serve as the proof term when judgemental cumulativity/equality is enough.


\subsection{noncanonical things}

\newcommand{\e}[1]{\texttt{-}#1}
\newcommand{\R}[2]{#1\texttt{:}#2}

We have variables $x$ within $t$ for some $\B xt$. We also have $e\e s$ for noncanonical $s$, an eliminator appropriate to the type of $e$ (e.g., if $e$ is a function, $s$ is its argument).

Finally, we have $\R tT$, a \emph{radical}, allowing us to annotate a canonical term with a canonical type, and thus form redexes.


\section{typing awaiting computation}

\newcommand{\ch}[2]{#1 \:\ni\: #2}
\newcommand{\sy}[2]{#1 \:\in\: #2}
\newcommand{\hy}{\;\vdash\;}
\newcommand{\CU}[2]{\lsq\A{Le}\:#1\:#2\rsq}

We shall have judgement forms for checking $\ch Tt$ (\emph{relying} on $\ch\TYPE T$) and synthesis $\sy eS$ (guaranteeing that $\ch\TYPE S$). At some point, these will be closed appropriately under computation, but let's leave that to the next section.


\subsection{type checking}

For sorts $w$, $u$, we have

\[
\Rule{w > u} {\ch wu}\qquad
\Axiom{\TYPE > \Type n}\quad
\Axiom{\TYPE > \Prop}\quad
\Rule{m > n}{\Type m > \Type n}\quad
\Axiom{\Type n > \Prop}
\]
I'm aware that the Coq tradition (from the days of impredicative \textbf{Set}) is to place $\Prop$ on a par with $\Type 0$ and below $\Type 1$, but the above formulation says yes to all those things and more. Besides, this system isn't called LEOG for nothing.

Meanwhile, for function types,
\[
\Rule{\ch wS \hg \sy xS \hy \ch uT}
     {\ch u \PI SxT}
     \;\mbox{where}\;
     w = \left\{\begin{array}{ll}\TYPE & \mbox{if}\;u=\Prop\\
                                     u & \mbox{otherwise}\end{array}\right.
\]
making $\Prop$ impredicative.

For functions, we have
\[\Rule{\sy xS \hy \ch Tt}{\ch{\PI SxT}{\B xt}}
\]

On the checking side, that leaves us with only the shipment of synthesizable terms.
\[
\Rule{e\in S \hg \CU ST \ni q}
     {T\ni \E eq }
\]
where this $\A{Le}$ thing is \emph{propositional cumulativity}. For all sorts $u$,
because for $u=\Prop$
\[
\Rule{\TYPE\ni S \hg \TYPE\ni T}
     {u\ni\CU ST}
\]
Let us have
\[
\Axiom{\CU\Prop{\Type n} \ni \nil}\qquad
\Rule{i < j}
     {\CU{\Type i}{\Type j}\ni \nil}
\]
and the (domain contravariant!)
\[
\Rule{\CU{S'}S\ni \nil \hg \sy x{S'}\hy \CU T{T'}\ni \nil}
{\CU{\PI SxT}{\PI{S'}x{T'}}\ni\nil}
\]
along with the inclusion of judgemental equality
\[
\Rule{\TYPE\ni S\equiv T}
{\CU ST \ni \nil}
\]
Nontrivial proofs of cumulativity will arrive in due course.


\subsection{type synthesis}

I write a reverse turnstile to indicate an appeal to a local hypothesis,
which is as close as I get to mentioning the context. Radicals are
straightforward, thanks to the `topsort'.
\[
\Rule{\dashv\; \sy xS}
{\sy x S}
\qquad
\Rule{\ch\TYPE T\hg \ch Tt}
     {\sy{\R tT}T}
\qquad
\Rule{\sy e{\PI SxT}\hg \ch Ss}
{\sy{e\e s}{T[\R sS]}}
\]
Application is unremarkable,
except to note that (i) there is no need to mention $x$ by name when
substituting it, because it is clear that it is $x$ which has moved out of
scope, and (ii) that $s$ has to be radicalised before it can be substituted
for $x$, from sheer syntactic compatibility and to create new redexes.



\end{document}
